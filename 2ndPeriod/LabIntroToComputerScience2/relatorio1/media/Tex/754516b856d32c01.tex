\documentclass[preview]{standalone}

\usepackage[english]{babel}
\usepackage[utf8]{inputenc}
\usepackage[T1]{fontenc}
\usepackage{lmodern}
\usepackage{amsmath}
\usepackage{amssymb}
\usepackage{dsfont}
\usepackage{setspace}
\usepackage{tipa}
\usepackage{relsize}
\usepackage{textcomp}
\usepackage{mathrsfs}
\usepackage{calligra}
\usepackage{wasysym}
\usepackage{ragged2e}
\usepackage{physics}
\usepackage{xcolor}
\usepackage{microtype}
\DisableLigatures{encoding = *, family = * }
\linespread{1}

\begin{document}

\begin{center}
Aqui é apresentado o gráfico de tempo de processamento dos métodos bubble, insertion e merge de ordenação para tamanhos de vetor de 25, 100, 1000 e 10000. Entretanto, é difícil analisar a eficiência dos algoritmos com gráficos de pouca profundidade. Assim, vamos aumentar o número de pontos para 20, de 0 a 10000, com espaçamento igual.
\end{center}

\end{document}
