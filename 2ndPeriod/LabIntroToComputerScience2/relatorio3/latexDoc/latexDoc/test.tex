%%%%%% Configurando os pacotes e comandos %%%%%%
%%%%%% Pule para a linha 62 %%%%%%%
\documentclass[fontsize=11pt]{article}
\usepackage[margin=0.70in]{geometry}
\usepackage{lipsum,mwe,abstract}
\usepackage[T1]{fontenc} 
\usepackage[brazilian]{babel} 
\usepackage{setspace}
\usepackage{caption}
\usepackage[hidelinks]{hyperref}
\usepackage{multirow}

\usepackage{fancyhdr} % Custom headers and footers
%\pagestyle{fancyplain} % Makes all pages in the document conform to the custom headers and footers
%\fancyhead{} 
%\fancyfoot[C]{\thepage} % Page numbering for right footer
\setlength\parindent{0pt} 
\setstretch{1.5}

\usepackage{amsmath,amsfonts,amsthm} % Math packages
\usepackage{wrapfig}
\usepackage{graphicx}
\usepackage{float}
\usepackage{subcaption}
\usepackage{comment}
\usepackage{enumitem}
\usepackage{cuted}
\usepackage{sectsty} % Allows customizing section commands
% \allsectionsfont{\normalfont \large \scshape} % Section names in small caps and normal fonts

\renewenvironment{abstract} % Change how the abstract look to remove margins
 {\small
  \begin{center}
  \bfseries \abstractname\vspace{-.5em}\vspace{0pt}
  \end{center}
  \list{}{%
    \setlength{\leftmargin}{0mm}
    \setlength{\rightmargin}{\leftmargin}%
  }
  \item\relax}
 {\endlist}
 
\makeatletter
\renewcommand{\maketitle}{\bgroup\setlength{\parindent}{0pt}% Change how the title looks like
\begin{center}
    \textbf{
      Universidade de São Paulo\\
      Instituto de Ciências Matemáticas e Computação
    }
\end{center}
\begin{flushleft}
  \textbf{\@title}
  \@author\\
  [3pt] 
  \@date
\end{flushleft}\egroup
}
\makeatother
%%% Daqui pra cima é apenas configuração %%%
%%%%%%%%%%%%%%%%%%%%%%%%%%%%%%%%%%%%%%%%%%%%

%%%%%% Definindo seus dados %%%%%%
\title{
\Large{Relatório 3 de Laboratório de Introdução à Ciências da Computação 2}\\
[10pt] 
}
\author{Vítor Amorim Fróis} 
\date{\today}
%%%%%%%%%%%%%%%%%%%%%%%%%%%%%%%%%%%

%%%%%% Iniciando seu relatório %%%%%% 
\begin{document}
\maketitle
 
\begin{abstract}
    

    %%%%%% Contextualize o seu trabalho. %%%%%%%
\end{abstract}

\rule{\linewidth}{0.2pt}

\section{Introdução}
    %%%%%% Faça a introdução do seu relatório. O que será feito? %%%%%%
    Durante o 3º módulo de LICC2, foram estudados métodos que alcançam tempo de
    ordenação linear. São eles o Counting, Bucket e Radix Sort.

\section{Metodologia e desenvolvimento}
    %%%%%% Explique a metodologia utilizada e o desenvolvimento do projeto. Como você obteve seus dados? Coloque seus códigos e embasamentos teóricos aqui se necessário. %%%%%%

\section{Resultados}
    %%%%%% Mostre os resultados obtidos através dos cálculos. Utilize imagens se necessário. %%%%%%
    
\section{Conclusão}
    %%%%%% Conclusão do relatório. O que você aprendeu nessa tarefa? %%%%%%


\bibliographystyle{plain}
\bibliography{references.bib}
    %%%%%% Lembre-se de adicionar as referências bibliográficas utilizadas no arquivo 'references.bib'e depois cita-las nessa seção. Conulta: https://pt.overleaf.com/learn/latex/Bibliography_management_in_LaTeX %%%%%%

\end{document}

 