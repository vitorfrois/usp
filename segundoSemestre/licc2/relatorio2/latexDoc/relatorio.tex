%%%%%% Configurando os pacotes e comandos %%%%%%
%%%%%% Pule para a linha 62 %%%%%%%
\documentclass[fontsize=11pt]{article}
\usepackage[margin=0.70in]{geometry}
\usepackage{lipsum,mwe,abstract}
\usepackage[T1]{fontenc} 
\usepackage[brazilian]{babel} 
\usepackage{setspace}
\usepackage{caption}
\usepackage[hidelinks]{hyperref}
\usepackage{multirow}

\usepackage{fancyhdr} % Custom headers and footers
%\pagestyle{fancyplain} % Makes all pages in the document conform to the custom headers and footers
%\fancyhead{} 
%\fancyfoot[C]{\thepage} % Page numbering for right footer
\setlength\parindent{0pt} 
\setstretch{1.5}

\usepackage{amsmath,amsfonts,amsthm, amssymb} % Math packages
\usepackage{wrapfig}
\usepackage{graphicx}
\usepackage{float}
\usepackage{subcaption}
\usepackage{comment}
\usepackage{enumitem}
\usepackage{cuted}
\usepackage{sectsty} % Allows customizing section commands
% \allsectionsfont{\normalfont \large \scshape} % Section names in small caps and normal fonts
\usepackage{biblatex}
\addbibresource{references.bib}

%code things
\usepackage{xcolor}
\usepackage{listings}

\lstdefinestyle{customc}{
  belowcaptionskip=1\baselineskip,
  breaklines=true,
  frame=L,
  xleftmargin=\parindent,
  language=C,
  showstringspaces=false,
  basicstyle=\small\ttfamily,
  keywordstyle=\bfseries\color{green!40!black},
  commentstyle=\itshape\color{purple!40!black},
  identifierstyle=\color{blue},
  stringstyle=\color{orange},
}
%%%

\definecolor{codegreen}{rgb}{0,0.6,0}
\definecolor{codegray}{rgb}{0.5,0.5,0.5}
\definecolor{codepurple}{rgb}{0.58,0,0.82}
\definecolor{backcolour}{rgb}{0.95,0.95,0.92}

\lstdefinestyle{mystyle}{
  backgroundcolor=\color{white},   
  commentstyle=\color{codegreen},
  keywordstyle=\color{magenta},
  numberstyle=\tiny\color{codegray},
  stringstyle=\color{codepurple},
  basicstyle=\linespread{0.9}\ttfamily\footnotesize,
  breakatwhitespace=false,         
  breaklines=true,                 
  captionpos=b,                    
  keepspaces=true,                 
  numbers=left,                    
  numbersep=5pt,                  
  showspaces=false,                
  showstringspaces=false,
  showtabs=false,                  
  tabsize=2,
  language=C
}
\lstset{escapechar=@,style=mystyle}
  
  
  \renewenvironment{abstract} % Change how the abstract look to remove margins
  {\small
  \begin{center}
  \bfseries \abstractname\vspace{-.5em}\vspace{0pt}
  \end{center}
  \list{}{%
    \setlength{\leftmargin}{0mm}
    \setlength{\rightmargin}{\leftmargin}%
  }
  \item\relax}
 {\endlist}
 
\makeatletter
\renewcommand{\maketitle}{\bgroup\setlength{\parindent}{0pt}% Change how the title looks like
\begin{center}
    \textbf{
      Universidade de São Paulo\\
      Instituto de Ciências Matemáticas e Computação
    }
\end{center}
\begin{flushleft}
  \textbf{\@title}
  \@author\\
  [3pt] 
  \@date
\end{flushleft}\egroup
}
\makeatother
%%% Daqui pra cima é apenas configuração %%%
%%%%%%%%%%%%%%%%%%%%%%%%%%%%%%%%%%%%%%%%%%%%

%%%%%% Definindo seus dados %%%%%%
\title{
\Large{Relatório 3 de Laboratório de Introdução à Ciências da Computação 2}\\
[10pt] 
}
\author{Vítor Amorim Fróis} 
\date{\today}
%%%%%%%%%%%%%%%%%%%%%%%%%%%%%%%%%%%

%%%%%% Iniciando seu relatório %%%%%% 
\begin{document}
\maketitle

\begin{abstract}
    %%%%%% Contextualize o seu trabalho. %%%%%%%

\end{abstract}

\rule{\linewidth}{0.2pt}

\section{Introdução}
    %%%%%% Faça a introdução do seu relatório. O que será feito? %%%%%%
    Durante o módulo 1, métodos de ordenação simples, como o Bubble, Insertion
    e Merge. Os dois primeiros tinham complexidade $O(n^2)$, enquanto o Merge 
    conseguia alcançar complexidade linear-logarítmica.
    Agora, através de novas ideias, é preciso aprimorar os métodos para buscar
    complexidade $O(n\log n)$ corriqueiramente. Esses são Heap e Quick. 

\section{Metodologia e desenvolvimento}
  \subsection{Heap Sort}
    O Heap Sort é um método de ordenação que usa a estrutura chamada de Heap.
    Essa é uma estrutura semelhante a uma árvore binária, mas também pode ser 
    representada através de um vetor. 
    \\ No algoritmo, os elementos são posicionados em max heap, operação que possui 
    complexidade $O(\log n)$. 
    \lstinputlisting[linerange={39-55}]{sorts.c}
    Então, troca os elementos $[1];[n]$ de posição e descarta
    o último elemento ao diminuir o tamanho $n$ do vetor. 
    \lstinputlisting[linerange={60-63}]{sorts.c}
    Novas chamadas 
    recursivas são executadas até que o vetor esteja completamente ordenado.
    Como é necessário passar pelo vetor uma vez para executar esse método,
    a complexidade é $O(n) \times O(\log n) = O(n\log n)$.
    Já que independente dos casos o mesmo algoritmo sempre é executado, 
    o pior caso é igual ao maior caso.
    A complexidade de espaço é $O(1)$, apenas para os \textit{swaps}.

  \subsection{Quick Sort}
    Já o Quick Sort é um algoritmo de ordenação muito eficiente que utiliza pivôs para ordenação. 
    A partir desse elemento, chamadas recursivas onde cada elemento deve ser menor que o pivô 
    são feitas e a ordenação se completa.
    Para o algoritmo manter sua alta eficiência, é necessário que a escolha de pivô seja bem 
    feita. Assim, uma técnica utilizada é fazer a mediana entre o primeiro, último e elemento médio
    do vetor. Isso busca garantir que o elemento escolhido esteja o mais próximo da média de valores 
    sem muito poder computacional.
    \lstinputlisting[linerange={82-91}]{sorts.c}
    No pior caso, as chamadas recursivas são do tipo $n-1$, e assim o algoritmo deve percorrer
    todo o vetor, elemento a elemento, para ordenar.
    Contudo, no melhor caso possível, há chamadas recursivas de tamanho $\dfrac{n}{2} - 1$ 
    (considere $n$ o tamanho da chamada recursiva anterior), de forma
    que o melhor caso executa $\log_2 n$ chamadas \cite{moacir}.
    \lstinputlisting[linerange={92-127}]{sorts.c}
    Pela análise do algoritmo, cada chamada possui complexidade $O(n)$. Assim, no melhor caso, 
    o algoritmo faz $\log n$ chamadas de $O(n)$, resultando em complexidade de tempo $O(n\log n)$.
    Um ponto importante de se destacar é a não utilização de vetores auxiliares, que 
    traz complexidade de espaço 0 para o algoritmo.
  \subsection{Comparação com métodos do módulo 1}  
    Apesar de compartilharem complexidade com Merge, os métodos estudados 
    no módulo presente são melhores que Insertion e Bubble. Assim, no caso médio,
    espera-se que log-lineares se sobressaiam, enquanto aqueles de ordem 
    quadrática performem pior. Ainda assim, cada método, mesmo que com piores
    constantes, pode ser útil em ocasiões especiais.
\section{Resultados}
  %%%%%% Mostre os resultados obtidos através dos cálculos. Utilize imagens se necessário. %%%%%%
  Para comparar Heap e Quick, serão plotados inicialmente gráficos até 10k e 100k.
  Então, um gráfico especial para o Quick será produzido a fim de aprofundar
  no pior caso desse algoritmo, que segundo a parte de Metodologia, tem 
  complexidade $O(n^2)$.
  \subsection{Caso Médio}
    É esperado que ao \textit{plottar} os gráficos
    das funções, os comportamentos sejam parecidos, já que a complexidade é igual,
    ao menos no caso médio.
    \begin{figure}[H]
      \includegraphics[width=\textwidth]{graph10k.png} 
      \caption{Ordenação de vetores randômicos com até 10000 elementos}
      \label{fig:10k}
    \end{figure}
    A primeira coisa que chama atenção nesse gráfico é a disparidade entre o Heap
    e o Quick. Isso se deve ao fato de que as constantes do primeiro são muito 
    maiores que as do segundo. Ainda é possível notar que apesar de parecer uma reta, 
    o gráfico possui uma leve inclinação, originária do $\log_2 n$ na função de 
    complexidade.
    \\ Com o tempo, a função logarítimica tende a encontrar uma inclinação menor, 
    portanto quanto maior o valor de $n$, mais $O(n \log_2 n)$ se aproxima, proporcionalmente,
    de uma função linear. Observar o gráfico com valores até 100000 
    permite melhor visualização.
    \begin{figure}[H]
      \includegraphics[width=\textwidth]{graph100k.png} 
      \caption{Ordenação de vetores randômicos com até 100000 elementos}
      \label{fig:100k}
    \end{figure}
  \subsection{Pior Caso}
    Entre os dois algoritmos estudados, somente o Quick apresenta pior
    caso. Assim, o gráfico busca comparar seu pior caso com o caso médio
    do Heap.
    \lstinputlisting[linerange={130-131}]{sorts.c}
    O pivo será escolhido como o primeiro elemento do vetor, o qual estará
    inversamente ordenado.
    \begin{figure}[H]
      \includegraphics[width=\textwidth]{quickWorse.png} 
      \caption{Pior caso do Quick Sort}
      \label{fig:worse}
    \end{figure}
    Assim, pode se observar que em seu pior caso, o Quick Sort realmente 
    apresenta complexidade $O(n^2)$, enquanto o Heap continua com pior caso
    $O(n \log n)$.
  \subsection{Comparação com métodos anteriores}
    Ao comparar com os métodos anteriores, o gráfico do caso médio fica como esperado
    \begin{figure}[H]
      \includegraphics[width=\textwidth]{allgraphs.png} 
      \caption{Todos métodos de ordenação}
      \label{fig:all}
    \end{figure}

  \section{Conclusão}
  %%%%%% Conclusão do relatório. O que você aprendeu nessa tarefa? %%%%%%
  Concluímos ao longo do relatório que os dois algoritmos são muito eficientes,
  porém, o Heap é tem uma garantia maior de complexidade. Para usar o Quick Sort,
  é essencial garantir que o pivô escolhido seja bom, caso contrário sua 
  complexidade dispara, tornando-se igual ao Insertion ou Bubble.
  \\ Ainda assim, cada método possui sua particularidade, e analisar um problema
  e suas nuances antes de escolher um algoritmo para resolver é essencial.
\printbibliography[heading=bibintoc, title={Referências}]
    %%%%%% Lembre-se de adicionar as referências bibliográficas utilizadas no arquivo 'references.bib'e depois cita-las nessa seção. Conulta: https://pt.overleaf.com/learn/latex/Bibliography_management_in_LaTeX %%%%%%

\end{document}

 